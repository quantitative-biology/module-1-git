\documentclass[aspectratio=169]{beamer}
%% For 4:3 aspect ratio:
%% \documentclass{beamer}
\usepackage{../git-course}

\title[git-course]{An Introduction to \gh\ and \gs}
\author{Andrew Edwards}
\date{\today}

\begin{document}
%% Needed to remove 'Figure:' from figure captions:
\setbeamertemplate{caption}{\raggedright\insertcaption\par}

\frame[plain]{
\titlepage
}

\section{Definitions}
\frame{\frametitle{Definitions}
  \bi
  \item Repository -- essentially a directory containing all your files for a
    project (plus some files that Git uses).
  \item Git -- a program that allows you to efficiently save ongoing versions of
    your files (`version control').
 \item GitHub -- a website that hosts your repositories so that you can easily
   share code and collaborate with colleagues.
  \ei

Basically, you work on your files in a repository on your computer, use Git on
your computer when you are happy with some changes, and share the files easily on GitHub.
}


\section{Contents}

\frame{\frametitle{Contents}
\bn
\item Creating -- create a new repository on \gh
\item Cloning -- copying it to your local computer
\item Committing -- the crux of working with Git
\en
  \bigskip
We will demonstrate the basics in this video lecture, which includes two
exercises for you.
}

\section{Creating}
\frame{\frametitle{Creating a new repository}
  \bi
    \item Sign into your \gh\ account, click on the
      \emph{Repositories} tab, and press the \emph{New} button.
    \item Give your repository a name. Let's call it \lst{test}.
    \item Check \emph{Initialize this repository with a README}.
    \item Leave \emph{`Add .gitignore`} and \emph{`Add a license`}
      set to \emph{None}.
    \item Click \emph{Create repository}.
  \ei

You now have a new repository on the \gh\ website. Next we will \lst{clone} it
onto your computer.
}

\section{Cloning}
\frame{\frametitle{Cloning your new repository}
  \bi
    \item Copy the full URL (web address) of your test repository.
    \item Open the \gs\ and navigated to your \lst{C:/github} directory
      (or whatever you called it when you created it in the setup instructions
      -- it's the place you are going to save all your Git repositories).
    \item run the following command to \emph{clone} your repository:

      ~~\lst{git clone URL}\\
      where \lst{URL} is the url of your newly created repository (paste should work).
  \ei
  ~\\
  You should now have a subdirectory called \lst{github/test} on your computer.\\

  In \gs, change
  to that directory:\\
  ~~\lst{cd test}
  ~\\

  So `clone' is Git speak for copying something from GitHub onto your local
  computer.
  ~\\
  This example has just one file (the README). But the process is the same for a
  repository with multiple files and multiple directories (the structure is
  fully preserved).
}

\frame{\frametitle{Windows only: Storing your credentials}
  When you are using the \gs\ for the \alert{very first time on Windows}, issue
  the following command:\\
  \bigskip
  \lst{git config --global credential.helper wincred}\\
  \bigskip
  This means that you don't have to repeatedly enter your \gh\ password (just do
  it when you are first prompted).\\
}

\section{Committing}
\begingroup
\small
\frame{\frametitle{Creating and committing a new file}
  \bi
    \item Create a new file, \emph{newFile.txt}, in the \lst{github/test}
      directory.
    \pause
    \item Add a line of text at the start of the file and save it.
    \pause
  \item Check the status of the (\lst{test}) repository by typing
    \lst{git status} in the Git shell.
    \pause
  \item It should say that you have an `Untracked file' called \emph{newFile.txt}.
      You want to tell Git to start tracking it, by using the \lst{git add}
      command:\\
      \lst{git add newFile.txt}
   \pause
 \item Type \lst{git status} again. (If on MacOS you may see mention of a
   \emph{.DS\_Store} file -- ignore that for now).
 \item You should see that the file is listed as a `new file` under `Changes to
   be committed`.
    \item Let's now `commit' it:\\
      \lst{git commit -a -m "Add newFile.txt file."}\\
      The commit message should be a useful message saying what the commit
      encapsulates.\\
    \pause
    \item Push the commit to \gh: \lst{git push}
    \pause
    \item Check (refresh) the \gh\ webpage and see your commit and the uploaded file.
  \ei
}
\endgroup

\frame{\frametitle{What just happened?}
  We just used three of the main Git commands:
  \bi
  \item \lst{git add <filename>} -- tell Git to start keeping track of changes
    to this file. You only need to tell Git this once.
  \item \lst{git commit -a -m "Message."} -- committing your changes, which means tell Git
    you are happy with your edits and want to save them. % You are actually saving
    % a snapshot of your entire repository (all the folders and files in your repository).
  \item \lst{git push} -- this sends your commit to the GitHub website.
  \ei

 You always have your files stored \emph{locally} on your computer (as usual),
 even if you don't \lst{add} them or \lst{commit} changes.

 ~\\

 When you push to GitHub then your colleagues can easily \lst{fetch} (retrieve) them.
}

\frame{\frametitle{Keyboard aliases (shortcuts)}
Now, \lst{git commit -a -m "Message."} is a bit much to type, so we have an
alias for it:

~\\

\lst{git com "Message."}

~\\

This is defined in the \emph{.gitconfig} file you installed in the `git-setup`
instructions.

~\\

The \lst{-a} means `commit all changes of files that Git is tracking`, and \lst{-m} is
to include a message. Since we usually want to do both of
these, \lst{git com "Message."} is a useful shortcut. But it is important to
realise it is an alias if searching online for help.

Similarly:

\lst{git s} -- for \lst{git status}

\lst{git p} -- for \lst{git push}

\lst{git d} -- for \lst{git diff}

From now on we will use the aliases.
}

\frame{\frametitle{Edit \emph{Readme.md}}
  Edit the \emph{Readme.md} file. Add some simple comments describing
  the project such as: ``A test repository for learning Git.''\\
  \bigskip
  \pause
  Look over the changes, commit them, and push them to your \gh\
  repository:\\
  \lst{git s}\\
  \lst{git diff} or the alias \lst{git d} -- this gives a simple look at the differences between the last committed
  version and your current version (of all files; only one in this case) \\
  \lst{git com "Initial edit of Readme.md"}\\
  \lst{git p}\\
  ~\\
  Refresh your \gh\ web page and you should see your text
  (the \emph{Readme.md} file is what is shown on the main page
  of your repo).
}

\frame{\frametitle{Exercise 1: create, edit, and commit \emph{simpleText.txt}}
  \bn
    \item Create a text file \lst{simpleText.txt}
      in your local \lst{test} repository. Add a line of text at the start and
      save it.
    \pause
    \item Predict what \lst{git s} will tell you, \emph{then} type it in the
      Git shell to check.\\
    \pause
    \item Add the file to the repository using the git commands:\\
      \lst{git add simpleText.txt}\\
      \lst{git s} -- not necessary but useful to check you understand what is
      changing before you commit\\
      \lst{git com "Adding simpleText.txt"}\\
      \lst{git p}
    \pause
    \item Add some text to \lst{simpleText.txt}, then \lst{git com "Message"} and \lst{git p}.
    \pause
    \item Repeat this a few times to get the hang of it, while intermittently doing \lst{git s} and
      \lst{git d} to understand what's changing.
    \pause
    \item Keep an eye on your commits by refreshing the \gh\ page.
  \en
}

\frame{\frametitle{Adding multiple files at once -- slide 1}
  Often you add multiple files in a new directory. When you run \lst{git s}, you
  will see a large list of \emph{Untracked files}. They can be added at once by simply
  adding the whole directory.
  \bi
    \item Create a new directory to your \lst{test} repository, using your normal method. Call it \lst{new-stuff}.
    \item Add a few new test files to that directory called
      \lst{test1.txt}, \lst{test2.txt}, etc. Put some example text
      in one or more of them if you want.
    \item On the command line, check the status:\\
      \lst{git s}
    \item You will see a listing showing the \lst{new-stuff} directory in
      \emph{Untracked files}.
%    \item To see the actual files to be committed instead of just the
%      directory:\\
%      \lst{git s -u}
    \item To add all the new files in preparation for a commit,
      issue the command:\\
      \lst{git add new-stuff/}
  \ei
  \bigskip
  Continued...
}

\frame{\frametitle{Adding multiple files at once -- slide 2}
  \bi
    \item Check the status of the repository again:
      \lst{git s}
    \item It will now show all files in \emph{Changes to be committed}
    \item Commit the changes:\\
      \lst{git com "Added new-stuff directory."}
    \item Push the changes to \gh:\\
      \lst{git p}
    \item Check your \gh\ webpage and see your commit and that the files
      have been uploaded.
    \item That works no matter how many files are in your \lst{new-stuff}
      directory.
      \ei
  Exercise 2: Try the above, to practice creating multiple files in a folder and
  committing that folder.
}

\frame{\frametitle{Adding multiple files at once -- slide 3}
  \bi
    \item To add multiple files with similar names you can use the wildcard
      \lst{*} symbol.
    \item You just added (told Git to keep track of) the new files in your
      \lst{new-stuff/} directory.
    \item If you add more new files to that directory, you will
      have to tell Git to track those also (since they are new -- you haven't
      told Git about them yet).
    \item Say you have 10 new files in \lst{new-stuff/} called \lst{idea1.txt}, \lst{idea2.txt}, ..., \lst{idea10.txt}.
    \item Instead of typing\\
      \lst{git add new-stuff/idea1.txt}\\
      \lst{git add new-stuff/idea2.txt}\\
      etc. (note the \lst{new-stuff/} folder name there)
      you can just use the wildcard \lst{*}:\\
      \lst{git add new-stuff/idea*.txt}\\
      (or even just \lst{git add new-stuff/*.txt}, or \lst{git add new-stuff/*.*}).
    \item No need to do this now, but this is useful to know.
  \ei
}


\frame{\frametitle{The \emph{.gitignore} file}
  But what if you don't want to add all the files that you create?

  ~\\

  Each repository can have a \emph{.gitignore} file, in the root directory
  of the repository.

  Such a file has names of files (such as \lst{my-secret-notes.txt}) or wildcard names (such as
  \lst{*.pdf} or \lst{*.doc}) that will be completely ignored by Git.

  ~\\

  When sharing a repository with others, you want to share your \emph{code} (for
  example, R or Python code) and maybe data, but generally \emph{not} share the output (such as figures
  that the code generates; more on this later). For reproducible research your colleague (or anyone)
  should be able to run your code to generate the results.

  ~\\

  Some programs you run may make temporary files that don't need to be tracked
  by Git, the names of which should also be included in your \emph{.gitignore}.
}

\frame{\frametitle{The \emph{.gitignore} file}
  When sharing code or collaborating you want to keep your
  repository as clean as possible and not clutter it up with files that other
  people don't need.

  ~\\

  So when you run \lst{git s} and see untracked files that you don't want to be
  tracked, add them (or a suitable wildcard expression) to your
  \emph{.gitignore} file so that they are not added inadvertently.

  ~\\

  This will also simplify your workflow (you don't need to keep being reminded
  that you have untracked files).

  ~\\
  If you are on MacOS and you find that folders have a \emph{.DS\_Store} file
  in them, then create (and add and commit) a \emph{.gitignore} file with \emph{.DS\_Store} as a line.
}

\frame{\frametitle{Thoughts/hints regarding commit messages}
  What to write in \lst{git com "Message"}?\\
  Ideally:
  \bi
    \item Want to describe \emph{what} and (sometimes \emph{why}) you did
       something.
    \item The \emph{how} is not needed since that will be explained by the
       actual changes in the code.
    \item Message should be informative for collaborators (including your future self).
  \ei
  {\red Bad}:\\
  ~~\lst{git com "Tweaked function."}\\
  {\blue Good}:\\
  ~~\lst{git com "Allow plot.biomass() to use extra colours."}\\
  ~\\
  A good rule of thumb is to complete the sentence "This commit will ...".
}


\section{Congratulations}
\frame{\frametitle{Git Workflow}
  You have now learnt the basics of using Git. By creating a public repository
  on \gh\ you can now release your code to the world!

  ~\\
  You can also choose the \emph{private repository} option when creating a
  repository, so that you can control who can see it.

  ~\\

  Next we will show how to collaborate with colleagues, which is where the
  usefulness of Git will become more apparent.
}

\end{document}
