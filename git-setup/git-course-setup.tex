\documentclass[aspectratio=169]{beamer}
%% For 4:3 aspect ratio:
%% \documentclass{beamer}
\usepackage{../git-course}

\title[git-course]{Setting up Git and \gh\ for the first time}
\author{Andrew Edwards}
\date{\course}

\begin{document}

\frame[plain]{
\titlepage
}

\frame{\frametitle{Getting started...}
  Before you start using Git you need to set up your computer to use it, and
  install a few other programs that are useful.

  ~\\

  This is a {\red one-time setup} and once it is done, you will be able to
  easily create new projects or join others in collaboration.

  ~\\

  We have tested the installations as much as feasible. If you have an issue
  then search the internet, as it may be due to some configuration on your
  particular computer.

  ~\\

  This module is for any operating system: Windows, MacOS, Linux or Unix.

  ~\\


}

\frame{\frametitle{What you will end up having installed}
  \bigskip
  These are programs/things you will need (instructions are on the next
  slides). Obviously skip any that you already have working.
  \bi
  \item A \gh\ account
  \item Git on your computer
  \item Diffmerge or something similar for comparing changes to files
  \item Markdown Pad 2 or Chrome extenstion or something similar for viewing Markdown
    files
%  \item R -- presumably a version not \emph{too} old (though it's hard for us to specify
%    exactly what version). Also a few R packages are given here
%  \item A way of running \LaTeX\ (such as Mik\TeX\ or Pro\TeX)
  \item A text editor that isn't Notepad
%  \item Pandoc (comes with RStudio)
  \ei
}

\frame{\frametitle{Get a \gh\ account}
  \bigskip
  \bi
    \item Sign up for \gh: \url{https://github.com}
    \item If possible, choose a user name that will make sense to colleagues,
      e.g. \textbf{andrew-edwards} or \textbf{cgrandin}, not
      \textbf{pink-unicorn}.
    \item Optional (but desirable): attach a photo (headshot) to your
      profile. This makes it easy for collaborators to identify you.
  \ei
}

% As such we will use a simple command shell (to type in commands)
%  rather than a GUI.

\frame{\frametitle{Install the Git application on your machine}
  \bi
    \item Windows:\\
      \url{https://github-windows.s3.amazonaws.com/GitHubSetup.exe}\\
      Note that some people have had issues trying to install on a DFO computer
      without having admin privileges. If so, you may have to use a different
      computer. It can also time out during installation due to a slow network
      connection, so you may have to do this stage at home.
    \item MAC:\\
      Type \emph{git} on the command line. You will either see that it
      is installed, or if you have OSX 10.9 or higher it will prompt you to
      install it. If your OSX is less than 10.9, or for any other reason, you
      can install git from here:\\
      \url{https://git-scm.com/download/mac}
    \item Linux:\\
      You'll need the same tools. Sorry, we don't have a machine to test things
      on, but if you're using Linux you presumably know how to install programs.
  \ei
}

\frame{\frametitle{Configure the Git application}
  \bi
    \item Windows:
      \bn
    \item Create a \lst{github} directory, such as
          \lst[escapechar="\\"]{C:\\github}. It is fine
          to put it in a different path, but make sure there are no spaces
          or special characters \emph{anywhere} in the full path.
        \item Open the GitHub Desktop Application. Press the gear icon and
          choose \emph{Options}.
        \item In \emph{Configure Git}, fill in your name and the
          email address you used for your GitHub account.
        \item Change your \emph{Clone Path} to \lst[escapechar="\\"]{C:\\github}
          (or whatever directory you created above).
        \item Make sure that for \emph{Default Shell}, \emph{PowerShell} is
          checked.
        \item Click \emph{Save} and close the application.
        \item Check the configuration by opening the \gs, (not the \gh\
          application). The directory that it starts in should be
          \lst[escapechar="\\"]{C:\\github} (or whatever directory you created
          above).
      \en
      \item MAC:
        \bn
          \item Create the directory \lst[escapechar="\~"]{\~/github}.
          \item Enjoy a beverage.
        \en
  \ei
}

\frame{\frametitle{Install the difftool}
  The difftool will be used to examine differences between different
  versions of files and also to simplify merging of branches and
  collaborator's code. There are many programs that can be used but
  for consistency we will use Diffmerge.\\
  \bigskip

  For both Windows and MAC, Install Diffmerge:\\
  \bigskip
  \url{https://sourcegear.com/diffmerge/downloads.php}\\
  \bigskip
  The configuration for directing git to use Diffmerge will be done in a
  later step.
}

\frame{\frametitle{Cloning the git-course repository}
  For these instructions, replace \lst{GITHUB-USER} with your
  \gh\ account name.
  \bi
    \item On the \gh\ webpage, sign into your account and navigate to:\\
      \url{https://github.com/pbs-assess/git-course}
    \item Press the \emph{Fork} button. This will give you a copy of the
      git-course repository.
    \item Navigate to your new forked repository:\\
      \lst{https://github.com/GITHUB-USER/git-course}
    \item Copy the URL for your repository.
    \item Windows: Open the \gs\ and run the following command to clone the
      repository, you can paste the URL into the \gs\ by pressing the right
      mouse button:\\
      \lst{git clone https://github.com/GITHUB-USER/git-course}
    \item You now have the files for the GitHub course on your computer.
    \item MAC: Open terminal and change to the github directory:\\
      \lst{cd ~/github}
      then run the clone command:\\
      \lst{git clone https://github.com/GITHUB-USER/git-course}
  \ei
}

\frame{\frametitle{Copy the \emph{.gitconfig} file}
  \bi
    \item Git uses a configuration file for your account info, name to use
      when committing, aliases for commands, and other things.
    \item Open up the \emph{content} sub-directory in the \emph{git-course}
      directory and copy the file \emph{.gitconfig}
    \item For Windows, copy this file (overwrite the existing file) to:\\
      \bigskip
      \lst[escapechar="\\"]{C:\\Users\\YOUR-COMPUTER-USER-NAME\\.gitconfig}\\
      \bigskip
      where \lst{YOUR-COMPUTER-USER-NAME} is your user name on your
      computer, not your \gh\ account name.
      \bigskip
    \item For MAC, copy this file (overwrite the existing file) to:\\
      \bigskip
      \lst[escapechar="\~"]{\~/.gitconfig}
      \bigskip
    \ei
}

\frame{\frametitle{Edit the \emph{.gitconfig} file}
  \bi
    \item Use your favourite editor to edit the file. Don't edit the one in the
      \emph{git-course/content} directory.
    \item Change the [user] settings to reflect your information.
    \item Change the [difftool] and [diffmerge] directories so they point to
      the location where you have DiffMerge.
    \item For Windows the location should be:\\
      \bigskip
      \lst[escapechar="\\"]{C:\\Program Files\\SourceGear\\Common\\DiffMerge\\sgdm.exe}
      \bigskip
    \item For MAC the location should be:\\
      \bigskip
      \lst{/usr/local/bin/diffmerge}
      \bigskip
    \ei
}

\frame{\frametitle{MAC only: make your output pretty}
  On the MAC, change to the \lst[escapechar="\~"]{\~/github} directory
  and run the following command:\\
  \bigskip
  \lst{git config --global color.ui.auto}\\
  \bigskip
  This will make your git output colored in a similar way to the Windows
  powershell version.
}

\frame{\frametitle{Markdown Pad}
  Each project has an associated \lst{README.md} file that appears
  on its \gh\ homepage. The extension \lst{.md} stands for Markdown
  and is just an ASCii text file that contains simple formatting (such as
  bold or italics). There are two options we have used to read markdown files,
  choose one:\\

  \bigskip

  \bi
    \item The Markdown Pad 2 editor/viewer which is easy to use:\\
      \url{http://markdownpad.com}. Just get the free version.
    \item The Chrome extension for markdown viewing:\\
      \url{https://chrome.google.com/webstore/detail/markdown-viewer/ckkdlimhmcjmikdlpkmbgfkaikojcbjk?hl=en}.
  \ei
}

\frame{\frametitle{R and packages installation}
  Some exercises rely on having R installed with several packages we'll be
  using. We are assuming everyone has R installed.\\
  Install the following packages in R (we'll need more during the workshop):\\

  \bigskip

  \bi
    \item \lst{knitr}
    \item \lst{xtable}
    \item \lst{devtools}
  \ei

  \bigskip

  Make sure the location of \lst{Rscript} is included in your
  \textbf{PATH} environment variable. This is necessary to use the batch
  files that build the slides using knitr/\LaTeX. To check if it is, type
  the following into the \gs\ or any other command-line window:\\
  \lst{Rscript --version}\\
  and you should see something like this:\\
  \bigskip
  R scripting front-end version 3.3.2 (2016-10-31)
}

\frame{\frametitle{\LaTeX\ installation -- Windows}
  You will need \LaTeX\ installed (because it will correctly format Research
  Documents, even though you won't need to learn \LaTeX). You will also
  be able to build all of these course presentations from code.
  \bi
  \item If you have Rtools installed you may already have Mik\TeX installed (see
        last bullet point below to test it).
  \item On Windows, if you have lots of disk space ($>1.6$Gb you can use)
        then we think it's
        best to install it using Pro\TeX since it includes all \LaTeX\ libraries:
        \url{https://ctan.org/pkg/protext}\\
   \item If disk space may be a problem then install Mik\TeX\ (you may just have
     to install further libraries during the workshop):
      \bigskip
      \url{https://miktex.org/download}\\
      \bigskip
   \item You can build this PDF (that you are reading) by entering the
      \lst{beamer/git-setup} directory and double-click the \emph{build-pdf.bat}
      file (close this PDF first) or type \lst{pdflatex git-course-setup}.
   \ei
}

\frame{\frametitle{\LaTeX\ installation -- Mac}
  \bi
  \item On MAC, install Mac\TeX:\\
      \bigskip
      \url{http://www.tug.org/mactex/}\\
      \bigskip
      Run the following two or three times (to ensure page numbers are accurate)
      in the \lst{beamer/git-setup} directory (close this PDF first):\\
      \lst{pdflatex git-course-setup.tex}
  \ei
  \bigskip
}

\frame{\frametitle{Text editor}
  You must have a text editor that is aware of outside changes in a file. This
  is necessary because if you have a file open in the editor and you download
  an updated version of the file, you want the editor to ask you if you want
  to use the updated version.\\

  \bigskip

  We know that \textbf{Emacs}, \textbf{Xemacs}, \textbf{vim} are okay, as is
  \textbf{RStudio} for \textbf{.r} (and other) files.

  \bigskip

  \textbf{Notepad} is not okay. But you can download and install
  \textbf{Notepad++} which is fine:

  \bigskip

  \url{https://notepad-plus-plus.org/download/v7.3.3.html}

  \bigskip

  [Andy uses \textbf{Xemacs} for everything (including running R) as it has many keyboard shortcuts
    and colours text and code based on the file extension; Chris uses a mixture
    of \textbf{Xemacs} and \textbf{RStudio}.]
}

\frame{\frametitle{Pandoc}
  \bigskip
  If you don't use \textbf{RStudio} you need to separately install
  \textbf{Pandoc} from \url{https://github.com/jgm/pandoc/releases/tag/2.2.3.2}.
  \bigskip

  This is used in the conversion from \textbf{Rmarkdown} code to \textbf{.pdf}.
  \bigskip

  We (Andy) found it best to choose `install for all users' then click
  `Advanced' before Install, as you can then choose the installation folder.
  Double check it gets added to your PATH.
}

\frame{\frametitle{If you work closely with others who are taking the course}
  We expect that are at least a couple of groups of people (say, from a Region)
  who work closely together within each group.\\
  \bigskip
  If there is code etc.~that you work on together (or plan to) then make sure
  one person has it and we will get you going with that in the course
  (especially as you'll be together but often work in different locations --
  when in the same room together it will be easy to plan how you should set
  up directory structure etc.).\\
  \bigskip
  Ideally the code could be made public, but if not we can work around this.\\
  \bigskip
  We will be providing example text and code to work on in the course.
}

\frame{\frametitle{To keep notes}
  Our slides are fairly detailed, but you may want to also take some notes.\\
  \bigskip
  If so, create a file \lst{git-course/myNotes.*} the usual way {\red on
    your computer}, where \lst{*} is \lst{txt},
  \lst{doc}, \lst{docx} or whatever you want to write in.\\
  \bigskip
  This file will remain private on your computer and not be shared (even if you
  end up making some amazing changes to our files that you want to share with
  everyone else).
  \bigskip
}

\frame{\frametitle{You're ready...}
  That's it! You should have everything set up to create new repositories, fork
  collaborator's repositories, commit changes, push your commits, create
  branches, and view differences. During the course you'll understand what
  all that means.\\
  \bigskip
  To get a sneak peek at what is to come, take a look at the git-course
  repository on \gh (and you should already have these files on your computer):\\
  \bigskip
  \url{https://github.com/pbs-assess/git-course}\\
  \bigskip
  In particular, the network graph which shows all our commits from the
  beginning:\\
  \bigskip
  \url{https://github.com/pbs-assess/git-course/network}\\
  \bigskip
  The network graph is an invaluable tool in keeping track of what is going
  on amongst you and you collaborators. We'll get into it more in the course.

  See you soon.
}

\end{document}
